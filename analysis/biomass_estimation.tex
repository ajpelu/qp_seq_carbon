\documentclass[]{article}
\usepackage{lmodern}
\usepackage{amssymb,amsmath}
\usepackage{ifxetex,ifluatex}
\usepackage{fixltx2e} % provides \textsubscript
\ifnum 0\ifxetex 1\fi\ifluatex 1\fi=0 % if pdftex
  \usepackage[T1]{fontenc}
  \usepackage[utf8]{inputenc}
\else % if luatex or xelatex
  \ifxetex
    \usepackage{mathspec}
  \else
    \usepackage{fontspec}
  \fi
  \defaultfontfeatures{Ligatures=TeX,Scale=MatchLowercase}
\fi
% use upquote if available, for straight quotes in verbatim environments
\IfFileExists{upquote.sty}{\usepackage{upquote}}{}
% use microtype if available
\IfFileExists{microtype.sty}{%
\usepackage{microtype}
\UseMicrotypeSet[protrusion]{basicmath} % disable protrusion for tt fonts
}{}
\usepackage[margin=1in]{geometry}
\usepackage{hyperref}
\hypersetup{unicode=true,
            pdftitle={Compute Biomass},
            pdfborder={0 0 0},
            breaklinks=true}
\urlstyle{same}  % don't use monospace font for urls
\usepackage{color}
\usepackage{fancyvrb}
\newcommand{\VerbBar}{|}
\newcommand{\VERB}{\Verb[commandchars=\\\{\}]}
\DefineVerbatimEnvironment{Highlighting}{Verbatim}{commandchars=\\\{\}}
% Add ',fontsize=\small' for more characters per line
\usepackage{framed}
\definecolor{shadecolor}{RGB}{248,248,248}
\newenvironment{Shaded}{\begin{snugshade}}{\end{snugshade}}
\newcommand{\AlertTok}[1]{\textcolor[rgb]{0.94,0.16,0.16}{#1}}
\newcommand{\AnnotationTok}[1]{\textcolor[rgb]{0.56,0.35,0.01}{\textbf{\textit{#1}}}}
\newcommand{\AttributeTok}[1]{\textcolor[rgb]{0.77,0.63,0.00}{#1}}
\newcommand{\BaseNTok}[1]{\textcolor[rgb]{0.00,0.00,0.81}{#1}}
\newcommand{\BuiltInTok}[1]{#1}
\newcommand{\CharTok}[1]{\textcolor[rgb]{0.31,0.60,0.02}{#1}}
\newcommand{\CommentTok}[1]{\textcolor[rgb]{0.56,0.35,0.01}{\textit{#1}}}
\newcommand{\CommentVarTok}[1]{\textcolor[rgb]{0.56,0.35,0.01}{\textbf{\textit{#1}}}}
\newcommand{\ConstantTok}[1]{\textcolor[rgb]{0.00,0.00,0.00}{#1}}
\newcommand{\ControlFlowTok}[1]{\textcolor[rgb]{0.13,0.29,0.53}{\textbf{#1}}}
\newcommand{\DataTypeTok}[1]{\textcolor[rgb]{0.13,0.29,0.53}{#1}}
\newcommand{\DecValTok}[1]{\textcolor[rgb]{0.00,0.00,0.81}{#1}}
\newcommand{\DocumentationTok}[1]{\textcolor[rgb]{0.56,0.35,0.01}{\textbf{\textit{#1}}}}
\newcommand{\ErrorTok}[1]{\textcolor[rgb]{0.64,0.00,0.00}{\textbf{#1}}}
\newcommand{\ExtensionTok}[1]{#1}
\newcommand{\FloatTok}[1]{\textcolor[rgb]{0.00,0.00,0.81}{#1}}
\newcommand{\FunctionTok}[1]{\textcolor[rgb]{0.00,0.00,0.00}{#1}}
\newcommand{\ImportTok}[1]{#1}
\newcommand{\InformationTok}[1]{\textcolor[rgb]{0.56,0.35,0.01}{\textbf{\textit{#1}}}}
\newcommand{\KeywordTok}[1]{\textcolor[rgb]{0.13,0.29,0.53}{\textbf{#1}}}
\newcommand{\NormalTok}[1]{#1}
\newcommand{\OperatorTok}[1]{\textcolor[rgb]{0.81,0.36,0.00}{\textbf{#1}}}
\newcommand{\OtherTok}[1]{\textcolor[rgb]{0.56,0.35,0.01}{#1}}
\newcommand{\PreprocessorTok}[1]{\textcolor[rgb]{0.56,0.35,0.01}{\textit{#1}}}
\newcommand{\RegionMarkerTok}[1]{#1}
\newcommand{\SpecialCharTok}[1]{\textcolor[rgb]{0.00,0.00,0.00}{#1}}
\newcommand{\SpecialStringTok}[1]{\textcolor[rgb]{0.31,0.60,0.02}{#1}}
\newcommand{\StringTok}[1]{\textcolor[rgb]{0.31,0.60,0.02}{#1}}
\newcommand{\VariableTok}[1]{\textcolor[rgb]{0.00,0.00,0.00}{#1}}
\newcommand{\VerbatimStringTok}[1]{\textcolor[rgb]{0.31,0.60,0.02}{#1}}
\newcommand{\WarningTok}[1]{\textcolor[rgb]{0.56,0.35,0.01}{\textbf{\textit{#1}}}}
\usepackage{longtable,booktabs}
\usepackage{graphicx}
% grffile has become a legacy package: https://ctan.org/pkg/grffile
\IfFileExists{grffile.sty}{%
\usepackage{grffile}
}{}
\makeatletter
\def\maxwidth{\ifdim\Gin@nat@width>\linewidth\linewidth\else\Gin@nat@width\fi}
\def\maxheight{\ifdim\Gin@nat@height>\textheight\textheight\else\Gin@nat@height\fi}
\makeatother
% Scale images if necessary, so that they will not overflow the page
% margins by default, and it is still possible to overwrite the defaults
% using explicit options in \includegraphics[width, height, ...]{}
\setkeys{Gin}{width=\maxwidth,height=\maxheight,keepaspectratio}
\IfFileExists{parskip.sty}{%
\usepackage{parskip}
}{% else
\setlength{\parindent}{0pt}
\setlength{\parskip}{6pt plus 2pt minus 1pt}
}
\setlength{\emergencystretch}{3em}  % prevent overfull lines
\providecommand{\tightlist}{%
  \setlength{\itemsep}{0pt}\setlength{\parskip}{0pt}}
\setcounter{secnumdepth}{0}
% Redefines (sub)paragraphs to behave more like sections
\ifx\paragraph\undefined\else
\let\oldparagraph\paragraph
\renewcommand{\paragraph}[1]{\oldparagraph{#1}\mbox{}}
\fi
\ifx\subparagraph\undefined\else
\let\oldsubparagraph\subparagraph
\renewcommand{\subparagraph}[1]{\oldsubparagraph{#1}\mbox{}}
\fi

%%% Use protect on footnotes to avoid problems with footnotes in titles
\let\rmarkdownfootnote\footnote%
\def\footnote{\protect\rmarkdownfootnote}

%%% Change title format to be more compact
\usepackage{titling}

% Create subtitle command for use in maketitle
\providecommand{\subtitle}[1]{
  \posttitle{
    \begin{center}\large#1\end{center}
    }
}

\setlength{\droptitle}{-2em}

  \title{Compute Biomass}
    \pretitle{\vspace{\droptitle}\centering\huge}
  \posttitle{\par}
    \author{}
    \preauthor{}\postauthor{}
    \date{}
    \predate{}\postdate{}
  

\begin{document}
\maketitle

\begin{Shaded}
\begin{Highlighting}[]
\KeywordTok{source}\NormalTok{(here}\OperatorTok{::}\KeywordTok{here}\NormalTok{(}\StringTok{"load_pkgs.R"}\NormalTok{))}
\end{Highlighting}
\end{Shaded}

\begin{Shaded}
\begin{Highlighting}[]
\NormalTok{df_tree <-}\StringTok{ }\KeywordTok{read_csv}\NormalTok{(here}\OperatorTok{::}\KeywordTok{here}\NormalTok{(}\StringTok{"data/tree_data_full.csv"}\NormalTok{), }\DataTypeTok{col_types =} \KeywordTok{cols}\NormalTok{())}

\CommentTok{# OJO Corección datos P028 // P024 // P024 }
\end{Highlighting}
\end{Shaded}

¿Qué especies tenemos en nuestros plots?

\begin{Shaded}
\begin{Highlighting}[]
\NormalTok{df_tree }\OperatorTok\StringTok{ }\KeywordTok{group_by}\NormalTok{(sp) }\OperatorTok\StringTok{ }\KeywordTok{count}\NormalTok{() }\OperatorTok\StringTok{ }\NormalTok{knitr}\OperatorTok{::}\KeywordTok{kable}\NormalTok{()}
\end{Highlighting}
\end{Shaded}

\begin{verbatim}
## group_by: one grouping variable (sp)
\end{verbatim}

\begin{verbatim}
## count: now 8 rows and 2 columns, one group variable remaining (sp)
\end{verbatim}

\begin{longtable}[]{@{}lr@{}}
\toprule
sp & n\tabularnewline
\midrule
\endhead
Adec & 1\tabularnewline
Aopa & 6\tabularnewline
Cmon & 36\tabularnewline
Pdul & 9\tabularnewline
Pter & 9\tabularnewline
Qilex & 3\tabularnewline
Qpyr & 3260\tabularnewline
Saria & 18\tabularnewline
\bottomrule
\end{longtable}

¿Cómo de puro o mixed es cada plot?

\begin{Shaded}
\begin{Highlighting}[]
\NormalTok{aux <-}\StringTok{ }\NormalTok{df_tree }\OperatorTok\StringTok{ }\KeywordTok{group_by}\NormalTok{(code, sp) }\OperatorTok\StringTok{ }
\StringTok{  }\KeywordTok{count}\NormalTok{() }\OperatorTok\StringTok{ }
\StringTok{  }\KeywordTok{spread}\NormalTok{(}\DataTypeTok{key =}\NormalTok{ sp, }\DataTypeTok{value =}\NormalTok{ n) }\OperatorTok\StringTok{ }
\StringTok{  }\KeywordTok{as.data.frame}\NormalTok{()}
\end{Highlighting}
\end{Shaded}

\begin{verbatim}
## group_by: 2 grouping variables (code, sp)
\end{verbatim}

\begin{verbatim}
## count: now 144 rows and 3 columns, 2 group variables remaining (code, sp)
\end{verbatim}

\begin{verbatim}
## spread: reorganized (sp, n) into (Adec, Aopa, Cmon, Pdul, Pter, …) [was 144x3, now 117x9]
\end{verbatim}

\begin{Shaded}
\begin{Highlighting}[]
\NormalTok{df <-}\StringTok{ }\NormalTok{aux }\OperatorTok\StringTok{ }
\StringTok{   }\CommentTok{# replace(is.na(.), 0) %>% }
\StringTok{   }\KeywordTok{mutate}\NormalTok{(}\DataTypeTok{n_total =} \KeywordTok{rowSums}\NormalTok{(dplyr}\OperatorTok{::}\KeywordTok{select}\NormalTok{(., }\OperatorTok{-}\NormalTok{code), }\DataTypeTok{na.rm =} \OtherTok{TRUE}\NormalTok{),}
          \DataTypeTok{per_pure =} \KeywordTok{round}\NormalTok{((Qpyr }\OperatorTok{/}\StringTok{ }\NormalTok{n_total)}\OperatorTok{*}\DecValTok{100}\NormalTok{, }\DecValTok{2}\NormalTok{))}
\end{Highlighting}
\end{Shaded}

\begin{verbatim}
## mutate: new variable 'n_total' with 55 unique values and 0% NA
\end{verbatim}

\begin{verbatim}
##         new variable 'per_pure' with 18 unique values and 0% NA
\end{verbatim}

Añadir datos de la parcela y filtrar por selected

\begin{Shaded}
\begin{Highlighting}[]
\NormalTok{dicc_plots <-}\StringTok{ }\KeywordTok{read_csv}\NormalTok{(here}\OperatorTok{::}\KeywordTok{here}\NormalTok{(}\StringTok{"raw_data/dicc_plots.csv"}\NormalTok{), }\DataTypeTok{col_types =} \KeywordTok{cols}\NormalTok{())}

\NormalTok{plot_s <-}\StringTok{ }\NormalTok{dicc_plots }\OperatorTok\StringTok{ }
\StringTok{  }\KeywordTok{inner_join}\NormalTok{(df, }\DataTypeTok{by =} \KeywordTok{c}\NormalTok{(}\StringTok{"original_code"}\NormalTok{ =}\StringTok{ "code"}\NormalTok{)) }\OperatorTok\StringTok{ }
\StringTok{  }\KeywordTok{filter}\NormalTok{(selected }\OperatorTok{==}\StringTok{ "TRUE"}\NormalTok{)}
\end{Highlighting}
\end{Shaded}

\begin{verbatim}
## inner_join: added 10 columns (Adec, Aopa, Cmon, Pdul, Pter, …)
\end{verbatim}

\begin{verbatim}
##             > rows only in x  (  0)
\end{verbatim}

\begin{verbatim}
##             > rows only in y  (  0)
\end{verbatim}

\begin{verbatim}
##             > matched rows     117
\end{verbatim}

\begin{verbatim}
##             >                 =====
\end{verbatim}

\begin{verbatim}
##             > rows total       117
\end{verbatim}

\begin{verbatim}
## filter: removed 12 rows (10%), 105 rows remaining
\end{verbatim}

Filtrar datos de los plots seleccionados, y ver que species tenemos

\begin{Shaded}
\begin{Highlighting}[]
\NormalTok{df_tree_sel <-}\StringTok{ }\NormalTok{df_tree }\OperatorTok\StringTok{ }
\StringTok{  }\KeywordTok{filter}\NormalTok{(code }\OperatorTok\StringTok{ }\NormalTok{plot_s}\OperatorTok{$}\NormalTok{original_code)}
\end{Highlighting}
\end{Shaded}

\begin{verbatim}
## filter: removed 183 rows (5%), 3,159 rows remaining
\end{verbatim}

\begin{Shaded}
\begin{Highlighting}[]
\NormalTok{df_tree_sel }\OperatorTok\StringTok{ }\KeywordTok{group_by}\NormalTok{(sp) }\OperatorTok\StringTok{ }\KeywordTok{count}\NormalTok{()}
\end{Highlighting}
\end{Shaded}

\begin{verbatim}
## group_by: one grouping variable (sp)
\end{verbatim}

\begin{verbatim}
## count: now 8 rows and 2 columns, one group variable remaining (sp)
\end{verbatim}

\begin{verbatim}
## # A tibble: 8 x 2
## # Groups:   sp [8]
##   sp        n
##   <chr> <int>
## 1 Adec      1
## 2 Aopa      6
## 3 Cmon     36
## 4 Pdul      8
## 5 Pter      5
## 6 Qilex     3
## 7 Qpyr   3082
## 8 Saria    18
\end{verbatim}

\hypertarget{componentes-de-la-biomasa}{%
\subsection{Componentes de la biomasa}\label{componentes-de-la-biomasa}}

\begin{itemize}
\tightlist
\item
  Stem with bark (commercial volume, up to a top diameter of 7 cm)
\item
  Thick branches (diameter greater than 7 cm)
\item
  Medium branches (diameter between 2 and 7 cm)
\item
  Thin branches (diameter smaller than 2 cm)
\item
  Leaves
\end{itemize}

Seguimos la misma nomenclatura que Ruiz-Peinado, Montero, and Del Rio
(2012)

\begin{itemize}
\tightlist
\item
  \(W_{s}\): Biomass weight of the stem fraction (kg)
\item
  \(W_{b7}\): Biomass weight of the thick branches fraction (diameter
  larger than 7 cm) (kg)
\item
  \(W_{b2-7}\): Biomass weight of medium branches fraction (diameter
  between 2 and 7 cm) (kg)
\item
  \(W_{b2+l}\): Biomass weight of thin branches fraction (diameter
  smaller than 2 cm) with leaves (kg)
\item
  \(W_{r}\): Biomass weight of the belowground fraction (kg)
\end{itemize}

\hypertarget{quercus-pyrenaica}{%
\subsection{\texorpdfstring{\emph{Quercus
pyrenaica}}{Quercus pyrenaica}}\label{quercus-pyrenaica}}

Fuente: Ruiz-Peinado, Montero, and Del Rio (2012)

\begin{itemize}
\tightlist
\item
  Stem + Thick branches \(W_{s} + W_{b7} = 0.0261 \cdot d^2 \cdot h\)
\item
  Medium branches
  \(W_{b2-7} = -0.0260 \cdot d^2 + 0.536 \cdot h + 0.00538 \cdot d^2 \cdot h\)
\item
  Thin branches \(W_{b2} = 0.898 \cdot d - 0.445 \cdot h\)
\item
  Roots \(W_{r} = 0.143 \cdot d^2\)
\end{itemize}

\begin{Shaded}
\begin{Highlighting}[]
\CommentTok{# Proposal function}
\NormalTok{biomassQpyr <-}\StringTok{ }\ControlFlowTok{function}\NormalTok{(d, h, ...)\{}
  
\NormalTok{  ws <-}\StringTok{ }\FloatTok{0.0261}\OperatorTok{*}\StringTok{ }\NormalTok{d}\OperatorTok{^}\DecValTok{2} \OperatorTok{*}\StringTok{ }\NormalTok{h}
\NormalTok{  wb7 <-}\StringTok{ }\OtherTok{NA} 
\NormalTok{  wb27 <-}\StringTok{ }\NormalTok{(}\OperatorTok{-}\FloatTok{0.0260} \OperatorTok{*}\StringTok{ }\NormalTok{d}\OperatorTok{^}\DecValTok{2}\NormalTok{) }\OperatorTok{+}\StringTok{ }\NormalTok{(}\FloatTok{0.536} \OperatorTok{*}\StringTok{ }\NormalTok{h) }\OperatorTok{+}\StringTok{ }\NormalTok{(}\FloatTok{0.00538} \OperatorTok{*}\StringTok{ }\NormalTok{d}\OperatorTok{^}\DecValTok{2} \OperatorTok{*}\StringTok{ }\NormalTok{h)}
\NormalTok{  wb2 <-}\StringTok{ }\NormalTok{(}\FloatTok{0.898}\OperatorTok{*}\NormalTok{d) }\OperatorTok{-}\StringTok{ }\NormalTok{(}\FloatTok{0.445}\OperatorTok{*}\NormalTok{h)}
\NormalTok{  wr <-}\StringTok{ }\FloatTok{0.143} \OperatorTok{*}\StringTok{ }\NormalTok{d}\OperatorTok{^}\DecValTok{2}
  
\NormalTok{  out <-}\StringTok{ }\KeywordTok{data.frame}\NormalTok{(ws, wb7, wb27, wb2, wr)}
  \KeywordTok{return}\NormalTok{(out)}
\NormalTok{\}}
\end{Highlighting}
\end{Shaded}

\hypertarget{computo-de-biomasa-para-qp}{%
\subsection{Computo de Biomasa para
Qp}\label{computo-de-biomasa-para-qp}}

\begin{Shaded}
\begin{Highlighting}[]
\NormalTok{df_qpyr <-}\StringTok{ }\NormalTok{df_tree_sel }\OperatorTok\StringTok{ }
\StringTok{  }\KeywordTok{filter}\NormalTok{(sp }\OperatorTok{==}\StringTok{ "Qpyr"}\NormalTok{) }\OperatorTok\StringTok{ }
\StringTok{  }\KeywordTok{filter}\NormalTok{(dbh }\OperatorTok{>}\StringTok{ }\DecValTok{7}\NormalTok{) }\OperatorTok\StringTok{ }
\StringTok{  }\KeywordTok{bind_cols}\NormalTok{(}\KeywordTok{map2_dfr}\NormalTok{(.}\OperatorTok{$}\NormalTok{dbh, .}\OperatorTok{$}\NormalTok{h, biomassQpyr)) }
\end{Highlighting}
\end{Shaded}

\begin{verbatim}
## filter: removed 77 rows (2%), 3,082 rows remaining
\end{verbatim}

\begin{verbatim}
## filter: removed 976 rows (32%), 2,106 rows remaining
\end{verbatim}

\begin{Shaded}
\begin{Highlighting}[]
\CommentTok{# tengo varios valores negativos en wb27 }
\NormalTok{df_qpyr }\OperatorTok\StringTok{ }\KeywordTok{mutate}\NormalTok{(}\DataTypeTok{wb27 =} \KeywordTok{ifelse}\NormalTok{(wb27 }\OperatorTok{<}\StringTok{ }\DecValTok{0}\NormalTok{, }\DecValTok{0}\NormalTok{, wb27)) }
\end{Highlighting}
\end{Shaded}

\begin{verbatim}
## mutate: changed 3 values (<1%) of 'wb27' (0 new NA)
\end{verbatim}

\begin{Shaded}
\begin{Highlighting}[]
\CommentTok{# Biomasa por plot }
\NormalTok{w_qpyr <-}\StringTok{ }\NormalTok{df_qpyr }\OperatorTok\StringTok{ }
\StringTok{  }\KeywordTok{group_by}\NormalTok{(code) }\OperatorTok\StringTok{ }
\StringTok{  }\KeywordTok{summarise_at}\NormalTok{(}\KeywordTok{vars}\NormalTok{(}\KeywordTok{starts_with}\NormalTok{(}\StringTok{"w"}\NormalTok{)), sum) }
\end{Highlighting}
\end{Shaded}

\begin{verbatim}
## group_by: one grouping variable (code)
\end{verbatim}

\begin{verbatim}
## summarise_at: now 105 rows and 6 columns, ungrouped
\end{verbatim}

??? Que hacemos con los mas pezqueñines??

\begin{Shaded}
\begin{Highlighting}[]
\NormalTok{df_qpyr7 <-}\StringTok{ }\NormalTok{df_tree_sel }\OperatorTok\StringTok{ }
\StringTok{  }\KeywordTok{filter}\NormalTok{(sp }\OperatorTok{==}\StringTok{ "Qpyr"}\NormalTok{) }\OperatorTok\StringTok{ }
\StringTok{  }\KeywordTok{filter}\NormalTok{(dbh }\OperatorTok{<}\StringTok{ }\DecValTok{7}\NormalTok{) }\OperatorTok\StringTok{ }
\StringTok{  }\KeywordTok{group_by}\NormalTok{(code) }\OperatorTok\StringTok{ }
\StringTok{  }\KeywordTok{count}\NormalTok{()}
\end{Highlighting}
\end{Shaded}

\begin{verbatim}
## filter: removed 77 rows (2%), 3,082 rows remaining
\end{verbatim}

\begin{verbatim}
## filter: removed 2,130 rows (69%), 952 rows remaining
\end{verbatim}

\begin{verbatim}
## group_by: one grouping variable (code)
\end{verbatim}

\begin{verbatim}
## count: now 36 rows and 2 columns, one group variable remaining (code)
\end{verbatim}

\hypertarget{quercus-ilex}{%
\subsection{\texorpdfstring{\emph{Quercus
ilex}}{Quercus ilex}}\label{quercus-ilex}}

Fuente: Ruiz-Peinado, Montero, and Del Rio (2012)

\begin{itemize}
\tightlist
\item
  Stem \(W_{s} = 0.143 \cdot d^2\)
\item
  Thick branches:

  \begin{itemize}
  \tightlist
  \item
    If \(d \leq 12.5\) then \(Z = 0\); If \(d > 12.5\) then \(Z = 1\)
  \item
    \(W_{b7} = [0.0684 \cdot (d - 12.5)^2 \cdot h ]\cdot Z\)
  \end{itemize}
\item
  Medium branches \(W_{b2-7} = 0.0898 \cdot d^2\)
\item
  Thin branches \(W_{b2+l} = 0.0824 \cdot d^2\)
\item
  Roots \(W_{r} = 0.254 \cdot d^2\)
\end{itemize}

Ojo aquellos ind que no tengan d \textgreater{} 12.5, no se considera la
fracción thick branches.

\begin{Shaded}
\begin{Highlighting}[]
\NormalTok{biomassQilex <-}\StringTok{ }\ControlFlowTok{function}\NormalTok{(d, h, ...)\{}
  \CommentTok{# aboveground biomass }
  \CommentTok{# nota: no he incluido la parte de thick branches }

\NormalTok{  ws <-}\StringTok{ }\FloatTok{0.143} \OperatorTok{*}\StringTok{ }\NormalTok{d}\OperatorTok{^}\DecValTok{2}
\NormalTok{  wb7 <-}\StringTok{ }\KeywordTok{ifelse}\NormalTok{(d }\OperatorTok{>}\StringTok{ }\FloatTok{12.5}\NormalTok{, }\FloatTok{0.0684} \OperatorTok{*}\StringTok{ }\NormalTok{((d }\OperatorTok{-}\StringTok{ }\FloatTok{12.5}\NormalTok{)}\OperatorTok{^}\DecValTok{2}\NormalTok{) }\OperatorTok{*}\StringTok{ }\NormalTok{h, }\OtherTok{NA}\NormalTok{)}
\NormalTok{  wb27 <-}\StringTok{ }\FloatTok{0.0898} \OperatorTok{*}\StringTok{ }\NormalTok{d}\OperatorTok{^}\DecValTok{2}
\NormalTok{  wb2 <-}\StringTok{  }\FloatTok{0.0824} \OperatorTok{*}\StringTok{ }\NormalTok{d}\OperatorTok{^}\DecValTok{2}
\NormalTok{  wr <-}\StringTok{ }\FloatTok{0.254} \OperatorTok{*}\StringTok{ }\NormalTok{d}\OperatorTok{^}\DecValTok{2}
  
\NormalTok{  out <-}\StringTok{ }\KeywordTok{data.frame}\NormalTok{(ws, wb7, wb27, wb2, wr)}
  \KeywordTok{return}\NormalTok{(out)}
\NormalTok{\} }
\end{Highlighting}
\end{Shaded}

\begin{Shaded}
\begin{Highlighting}[]
\NormalTok{df_qilex <-}\StringTok{ }\NormalTok{df_tree_sel }\OperatorTok\StringTok{ }
\StringTok{  }\KeywordTok{filter}\NormalTok{(sp }\OperatorTok{==}\StringTok{ "Qilex"}\NormalTok{) }\OperatorTok\StringTok{ }
\StringTok{  }\KeywordTok{bind_cols}\NormalTok{(}\KeywordTok{map2_dfr}\NormalTok{(.}\OperatorTok{$}\NormalTok{dbh, .}\OperatorTok{$}\NormalTok{h, biomassQilex))}
\end{Highlighting}
\end{Shaded}

\begin{verbatim}
## filter: removed 3,156 rows (>99%), 3 rows remaining
\end{verbatim}

\begin{Shaded}
\begin{Highlighting}[]
\NormalTok{w_qilex <-}\StringTok{ }\NormalTok{df_qilex }\OperatorTok\StringTok{ }
\StringTok{  }\KeywordTok{group_by}\NormalTok{(code) }\OperatorTok\StringTok{ }
\StringTok{  }\KeywordTok{summarise_at}\NormalTok{(}\KeywordTok{vars}\NormalTok{(}\KeywordTok{starts_with}\NormalTok{(}\StringTok{"w"}\NormalTok{)), sum) }
\end{Highlighting}
\end{Shaded}

\begin{verbatim}
## group_by: one grouping variable (code)
\end{verbatim}

\begin{verbatim}
## summarise_at: now one row and 6 columns, ungrouped
\end{verbatim}

\hypertarget{otras-especies}{%
\subsection{Otras especies}\label{otras-especies}}

\hypertarget{pistacia-therebinthus}{%
\subsubsection{Pistacia therebinthus}\label{pistacia-therebinthus}}

Fuente: Oyonarte and Cerrillo (2003) para datos de arbustos de Andalucía

\(y = 323.49 * x^{1.130}\)

siendo \(y\) la fitomasa, \(x\) el volumen. Se han calculado para
arbustos, determinando el volumen con dos morfotipos:

\begin{itemize}
\tightlist
\item
  \(V = 4/3\cdot \pi \cdot [(D_m)/2]^2 \cdot h\)
\item
  \(V = 4/3\cdot \pi \cdot [(D_m)/2] \cdot h\)
\end{itemize}

Limitaciones: - \(D_m\) es la media del Crown diameter mayor y menor -
Solamente aboveground biomass - No fraccionados los componentes de la
biomasa

\hypertarget{crataegus-monogyna}{%
\subsubsection{Crataegus monogyna}\label{crataegus-monogyna}}

Fuente: Oyonarte and Cerrillo (2003) para datos de arbustos de Andalucía

\(y = 1280.7 * x^{1.087}\)

siendo \(y\) la fitomasa, \(x\) el volumen. Se han calculado para
arbustos, determinando el volumen con el morfotipo:

\begin{itemize}
\tightlist
\item
  \(V = 1/3\cdot \pi \cdot [(D_m)/2]^2 \cdot h\)
\end{itemize}

Limitaciones: - \(D_m\) es la media del Crown diameter mayor y menor -
Solamente aboveground biomass - No fraccionados los componentes de la
biomasa

\hypertarget{adenocarpus-decorticans}{%
\subsection{Adenocarpus decorticans}\label{adenocarpus-decorticans}}

Fuente: Robles et al. (2006) para matorrales de Sierra Nevada

\(y = 6.60588.7 * x^{0.942512}\)

siendo \(y\) la fitomasa, \(x\) el volumen. Se han calculado para
arbustos, determinando el volumen con el morfotipo cilindro circular:

\begin{itemize}
\tightlist
\item
  \(V = \pi \cdot d^2 \cdot h/4\)
\end{itemize}

siendo \(d\) el diámetro (crow diameter)

\hypertarget{acer-opalus-granatensis}{%
\subsection{Acer opalus granatensis}\label{acer-opalus-granatensis}}

\hypertarget{prunus-dulcis-avium}{%
\subsection{Prunus dulcis / avium}\label{prunus-dulcis-avium}}

\hypertarget{sorbus-aria}{%
\subsection*{Sorbus aria}\label{sorbus-aria}}
\addcontentsline{toc}{subsection}{Sorbus aria}

\hypertarget{refs}{}
\leavevmode\hypertarget{ref-OyonarteCerrillo2003AbovegroundPhytomass}{}%
Oyonarte, P. Blanco, and R. Navarro Cerrillo. 2003. ``Aboveground
Phytomass Models for Major Species in Shrub Ecosystems of Western
Andalusia.'' \emph{Forest Systems} 12 (3): 47--55.
\url{https://doi.org/10.5424/1078}.

\leavevmode\hypertarget{ref-Roblesetal2006NineNative}{}%
Robles, A. B., P. Fern\a'andez, J. Ruiz-Mirazo, M. E. Ramos, C. B.
Passera, and J. L. Gonz\a'alez-Rebollar. 2006. ``Nine Native Leguminous
Shrub Species: Allometric Regression Equations and Nutritive Values.''
In \emph{Sustainable Grassland Productivity: Proceedings of the 21st
General Meeting of the European Grassland Federation, Badajoz, Spain,
3-6 April, 2006}, edited by J. Lloveras, A. Gonz\a'alez-Rodríguez, O.
V\a'azquez-Yañez, J. Piñeiro, O. Santamar\a'ıa, L. Olea, and M. J.
Poblaciones, 309--11. Madrid, Spain: Sociedad Española para el Estudio
de los Pastos (SEEP).

\leavevmode\hypertarget{ref-RuizPeinadoetal2012BiomassModels}{}%
Ruiz-Peinado, R., G. Montero, and M. Del Rio. 2012. ``Biomass Models to
Estimate Carbon Stocks for Hardwood Tree Species.'' \emph{Forest
Systems} 21 (1): 42. \url{https://doi.org/10.5424/fs/2112211-02193}.


\end{document}
